\documentclass{article}

\usepackage[utf8]{inputenc}
\usepackage{listings}
\usepackage{graphicx}
\usepackage{xcolor}
\usepackage{float}
\usepackage{hyperref}
\usepackage{longtable}

\hypersetup{
    colorlinks=true,
    linkcolor=blue,
    filecolor=magenta,      
    urlcolor=cyan,
    pdftitle={Overleaf Example},
    pdfpagemode=FullScreen,
    }

\urlstyle{same}

\definecolor{mGreen}{rgb}{0,0.6,0}
\definecolor{mGray}{rgb}{0.5,0.5,0.5}
\definecolor{mPurple}{rgb}{0.58,0,0.82}
\definecolor{backgroundColour}{rgb}{0.95,0.95,0.92}

\lstdefinestyle{CStyle}{
    backgroundcolor=\color{backgroundColour},   
    commentstyle=\color{mGreen},
    keywordstyle=\color{magenta},
    numberstyle=\tiny\color{mGray},
    stringstyle=\color{mPurple},
    basicstyle=\footnotesize,
    breakatwhitespace=false,         
    breaklines=true,                 
    captionpos=b,                    
    keepspaces=true,                 
    numbers=left,                    
    numbersep=5pt,                  
    showspaces=false,                
    showstringspaces=false,
    showtabs=false,                  
    tabsize=2,
    language=C
}
\lstset{style=CStyle}

\title{Personal Finance Tool Software Requirements Specification}
\author{Sean Twomey}

\begin{document}

\maketitle
\tableofcontents
\newpage

\section{Revision History}

\begin{longtable}{|p{2cm}|p{2cm}|p{5cm}|p{3cm}|}
    \hline
    \textbf{Revision} & \textbf{Date} & \textbf{Revision Description} & \textbf{Author} \\
    \hline
    \texttt{v1.0.0} & 03/01/2024 & Initial release of requirements set. & Sean Twomey \\
    \hline
    \caption{Table containing revision history of this document.}
\end{longtable}

\section{Acronym Table}

\begin{longtable}{|p{2cm}|p{10cm}|}
    \hline
    \textbf{Acronym} & \textbf{Full Description} \\
    \hline
    SRS & Software Requirements Specification \\
    \hline
    PFT & Personal Finance Tool \\
    \hline
    \caption{Table containing acronyms/abbreviations used throughout document and their full descriptions.}
\end{longtable}


\section{Introduction}

The Software Requirements Specification (SRS) should ultimately outline what this piece of software will do and how it will fulfill the needs of all stakeholders. This section will outline preliminary information about the Personal Finance Tool (PFT) to help introduce a user or stakeholder to the application. After
reading this section it should be clear the purpose of the application, who it's intended for use by, and what it should be used for.

\subsection{Purpose}

For \texttt{v1.0.0} the purpose of the PFT is to help a user understand their financial picture better with regards to the funds they have available, the upcoming bills they must pay, and the amount that they have leftover to spend as they see fit.
This application should allow a user to realize how much of their available money can be distributed to other spending purposes/goals as opposed to the portion of their funds that are tied up in meeting obligatory payments. \\
\\
This application is intended for use by anybody who has money available to them, reoccuring bills, and spending goals. Essentially anybody who wishes to differentiate between
the total funds they have available to them, the funds that are reserved for obligatory payments, and then the funds they have available for meeting their other spending goals. 

\subsection{Intended Audience}

This SRS is meant for use by developers, testers, product/project managers, sales and marketing individuals, or anybody else that would like to gain a better understanding of the requirements for the tool. After reading this document, a contributor or stakeholder should understand
the purpose of the PFT and functionalities the application possesses.

\subsection{Intended Use}

The primary use of this application will consist of on a month-by-month basis, a user managing their available funds throughout the month. They will enter into the application their available funds, their obligatory bills, whether or not the bills have been paid, and more.
The user will use the tool to understand how much money they have available, how much of that money needs to be allocated for upcoming bills, when these bills are due, and how much money is available for other spending purposes.
This application is meant for use by anyone who has money available to them, reoccurring bills, and wishes to understand their financial picture better.

\subsection{Product Scope}

The scope of the PFT is to take the form of a desktop-based applicaion which initially allow a user to input their total available funds and the bills they must pay with associated data. The tool will then allow a user to select which of the bills they have paid or have yet to pay thoughout the month.
The tool will compute the amount of money they have available for other spending purposes as the month goes on. It will also offer a financial picture to the user based on saved information the user has entered, allowing them to understand
their financial situation over time.

\subsection{Assumptions and Dependencies}

It's assumed that this application will be developed with a modern programming language and framework. It's also assumed that users of the application
will have access to a computer where they can run the application and enter their personal financial information. This application does not have any dependencies on anything other than the technical stack that will be used to create it.

\section{Prompt User For Financial Information On First Launch}

\begin{longtable}{|p{2cm}|p{6cm}| p{6cm}|}
    \hline
    \textbf{Requirement Number} & \textbf{Requirement Name} & \textbf{Requirement Text}\\
    \hline
    F1 & Prompt User For Financial Information On First Launch & The PFT shall prompt the user to enter their financial information the first time it is launched. \\
    \hline
\end{longtable}

\subsection{Prompt User For Total Available Money}

\begin{longtable}{|p{2cm}|p{6cm}| p{6cm}|}
    \hline
    \textbf{Requirement Number} & \textbf{Requirement Name} & \textbf{Requirement Text}\\
    \hline
    F1.1 & Prompt User For Total Available Money & The PFT shall initially prompt the user to enter the total amount of money they have upon first launch and then transition into prompting them for their bills. \\
    \hline
\end{longtable}


\subsubsection{Enforce A Positive Dollar Amount Is Entered}

\begin{longtable}{|p{2cm}|p{6cm}| p{6cm}|}
    \hline
    \textbf{Requirement Number} & \textbf{Requirement Name} & \textbf{Requirement Text}\\
    \hline
    F1.1.2 & Enforce A Positive Dollar Amount Is Entered & The PFT shall make sure that the total amount of money entered is a positive value to two decimal places. If a user enters a negative number, an appropriate error message will be generated and the user will need to re-attempt entering the value. \\
    \hline
\end{longtable}

\subsection{Prompt User For Bills}

\begin{longtable}{|p{2cm}|p{6cm}| p{6cm}|}
    \hline
    \textbf{Requirement Number} & \textbf{Requirement Name} & \textbf{Requirement Text}\\
    \hline
    F1.2 & Prompt User For Bills & The PFT shall prompt the user to enter their bills in the form of a payee, dollar amount owed and whether or not the bill has been paid following the request for the total amount of money available. \\
    \hline
\end{longtable}


\subsubsection{Enforce A Non-Empty Payee Is Entered}

\begin{longtable}{|p{2cm}|p{6cm}| p{6cm}|}
    \hline
    \textbf{Requirement Number} & \textbf{Requirement Name} & \textbf{Requirement Text}\\
    \hline
    F1.2.1 & Enforce A Non-Empty Payee Is Entered & The PFT shall make sure that the payee for each bill is not an empty string. If a user attempts to save a bill with an empty payee, an appropriate error message will be generated and the user will need to re-attempt entering the bill. \\
    \hline
\end{longtable}

\subsubsection{Enforce A Non-Empty And Positive Amount Owed Is Entered}

\begin{longtable}{|p{2cm}|p{6cm}| p{6cm}|}
    \hline
    \textbf{Requirement Number} & \textbf{Requirement Name} & \textbf{Requirement Text}\\
    \hline
    F1.2.2 & Enforce A Non-Empty And Positive Amount Owed Is Entered & The PFT shall make sure that the amount owed for each bill is not empty and is a positive value. If a user attempts to save a bill with an empty or negative amount owed, an appropriate error message will be generated and the user will need to re-attempt entering the bill. \\
    \hline
\end{longtable}

\subsection{Allow Users To Confirm Their Initial Financial Information}

\begin{longtable}{|p{2cm}|p{6cm}| p{6cm}|}
    \hline
    \textbf{Requirement Number} & \textbf{Requirement Name} & \textbf{Requirement Text}\\
    \hline
    F2 & Allow User To Confirm Their Initial Financial Information & The PFT shall allow the user to confirm their initial financial information when prompted on application start-up.  \\
    \hline
\end{longtable}

\subsubsection{Allow Users To Cancel Inputting Their Initial Financial Information}

\begin{longtable}{|p{2cm}|p{6cm}| p{6cm}|}
    \hline
    \textbf{Requirement Number} & \textbf{Requirement Name} & \textbf{Requirement Text}\\
    \hline
    F2.1 & Allow Users To Cancel Inputting Their Initial Financial Information & The PFT shall allow the user to cancel inputting their intial financial information using a "Cancel" control which will not save results to the configuration file and will close the application.  \\
    \hline
\end{longtable}

\subsubsection{Allow Users To Continue Inputting Their Initial Financial Information}

\begin{longtable}{|p{2cm}|p{6cm}| p{6cm}|}
    \hline
    \textbf{Requirement Number} & \textbf{Requirement Name} & \textbf{Requirement Text}\\
    \hline
    F2.2 & Allow Users To Continue Inputting Their Initial Financial Information & The PFT shall allow the user to continue inputting their intial financial information using a "Continue" control which will continue prompting the user to enter another bill.  \\
    \hline
\end{longtable}

\subsubsection{Allow Users To Save Their Initial Financial Information}

\begin{longtable}{|p{2cm}|p{6cm}| p{6cm}|}
    \hline
    \textbf{Requirement Number} & \textbf{Requirement Name} & \textbf{Requirement Text}\\
    \hline
    F2.2 & Allow User To Save Their Initial Financial Information & The PFT shall allow the user to save their initial financial information once finished entering it, using a "Save" control which will save the results to the configuration file.  \\
    \hline
\end{longtable}

\section{Display User's Financial Dashboard}

\begin{longtable}{|p{2cm}|p{6cm}| p{6cm}|}
    \hline
    \textbf{Requirement Number} & \textbf{Requirement Name} & \textbf{Requirement Text}\\
    \hline
    F3 & Display User's Financial Dashboard & The PFT shall use the configuration file to generate a dashboard which displays the user's financial information. \\
    \hline
\end{longtable}

\subsection{Display User's Total Money Available}

\begin{longtable}{|p{2cm}|p{6cm}| p{6cm}|}
    \hline
    \textbf{Requirement Number} & \textbf{Requirement Name} & \textbf{Requirement Text}\\
    \hline
    F3.1 & Display User's Total Money Available & The PFT shall display the saved total amount of money available to the user.  \\
    \hline
\end{longtable}

\subsection{Display User's Bills}

\begin{longtable}{|p{2cm}|p{6cm}| p{6cm}|}
    \hline
    \textbf{Requirement Number} & \textbf{Requirement Name} & \textbf{Requirement Text}\\
    \hline
    F3.2 & Display User's Bills & The PFT shall display the saved bills in the form of a payee and amount owed to the user.  \\
    \hline
\end{longtable}

\subsubsection{Display Whether Or Not Each Bill Needs To Be Paid}

\begin{longtable}{|p{2cm}|p{6cm}| p{6cm}|}
    \hline
    \textbf{Requirement Number} & \textbf{Requirement Name} & \textbf{Requirement Text}\\
    \hline
    F3.3 & Display Whether Or Not Each Bill Needs To Be Paid & The PFT shall display whether or not each bill needs to be paid.  \\
    \hline
\end{longtable}

\section{Allow User To Adjust Financial Dashboard}

\begin{longtable}{|p{2cm}|p{6cm}| p{6cm}|}
    \hline
    \textbf{Requirement Number} & \textbf{Requirement Name} & \textbf{Requirement Text}\\
    \hline
    F4 & Allow User To Adjust Financial Dashboard & The PFT shall allow the user to adjust their financial dashboard.  \\
    \hline
\end{longtable}

\subsection{Allow User To Adjust Total Money Available}

\begin{longtable}{|p{2cm}|p{6cm}| p{6cm}|}
    \hline
    \textbf{Requirement Number} & \textbf{Requirement Name} & \textbf{Requirement Text}\\
    \hline
    F4.1 & Allow User To Adjust Total Money Available & The PFT shall allow the user to adjust the total amount of money they have available.  \\
    \hline
\end{longtable}

\subsection{Allow User To Adjust Bills}

\begin{longtable}{|p{2cm}|p{6cm}| p{6cm}|}
    \hline
    \textbf{Requirement Number} & \textbf{Requirement Name} & \textbf{Requirement Text}\\
    \hline
    F4.2 & Allow User To Adjust Bills & The PFT shall allow the user to adjust the bills they have.  \\
    \hline
\end{longtable}

\subsubsection{Allow User To Add Bills}

\begin{longtable}{|p{2cm}|p{6cm}| p{6cm}|}
    \hline
    \textbf{Requirement Number} & \textbf{Requirement Name} & \textbf{Requirement Text}\\
    \hline
    F4.2.1 & Allow User To Add Bills & The PFT shall allow the user to create new bills that they have.  \\
    \hline
\end{longtable}

\subsubsection{Allow User To Edit Bills}

\begin{longtable}{|p{2cm}|p{6cm}| p{6cm}|}
    \hline
    \textbf{Requirement Number} & \textbf{Requirement Name} & \textbf{Requirement Text}\\
    \hline
    F4.2.2 & Allow User To Edit Bills & The PFT shall allow the user to edit bills that they have by modifying the payee and amount owed.  \\
    \hline
\end{longtable}

\subsubsection{Allow User To Remove Bills}

\begin{longtable}{|p{2cm}|p{6cm}| p{6cm}|}
    \hline
    \textbf{Requirement Number} & \textbf{Requirement Name} & \textbf{Requirement Text}\\
    \hline
    F4.2.3 & Allow User To Remove Bills & The PFT shall allow the user to remove pre-existing bills that they have created.  \\
    \hline
\end{longtable}

\subsubsection{Allow User To Edit Whether Or Not Each Bill Has Been Paid}

\begin{longtable}{|p{2cm}|p{6cm}| p{6cm}|}
    \hline
    \textbf{Requirement Number} & \textbf{Requirement Name} & \textbf{Requirement Text}\\
    \hline
    F4.2.4 & Allow User To Adjust Whether Or Not Each Bill Has Been Paid & The PFT shall allow the user to change whether or not each of their bills have been paid.  \\
    \hline
\end{longtable}

\subsection{Allow Users To Confirm Their Adjustments}

\begin{longtable}{|p{2cm}|p{6cm}| p{6cm}|}
    \hline
    \textbf{Requirement Number} & \textbf{Requirement Name} & \textbf{Requirement Text}\\
    \hline
    F5 & Allow User To Confirm Their Adjustments & The PFT shall allow the user to confirm their additions, edits and removals from the dashboard.  \\
    \hline
\end{longtable}

\subsubsection{Allow Users To Cancel Their Adjustments}

\begin{longtable}{|p{2cm}|p{6cm}| p{6cm}|}
    \hline
    \textbf{Requirement Number} & \textbf{Requirement Name} & \textbf{Requirement Text}\\
    \hline
    F5.1 & Allow User To Cancel Their Adjustments & The PFT shall allow the user to cancel their additions, edits and removals from the dashboard using a "Cancel" control which will not save results to the configuration file.  \\
    \hline
\end{longtable}

\subsubsection{Allow Users To Save Their Adjustments}

\begin{longtable}{|p{2cm}|p{6cm}| p{6cm}|}
    \hline
    \textbf{Requirement Number} & \textbf{Requirement Name} & \textbf{Requirement Text}\\
    \hline
    F5.2 & Allow User To Save Their Adjustments & The PFT shall allow the user to save their additions, edits and removals from the dashboard using a "Save" control which will save the results to the configuration file.  \\
    \hline
\end{longtable}

\section{Allow Users To Exit The PFT}

\begin{longtable}{|p{2cm}|p{6cm}| p{6cm}|}
    \hline
    \textbf{Requirement Number} & \textbf{Requirement Name} & \textbf{Requirement Text}\\
    \hline
    F6& Allow Users To Exit The PFT & The PFT shall allow the user to exit the application using an "X" control.  \\
    \hline
\end{longtable}

\section{Handle Configuration File Existence}

\begin{longtable}{|p{2cm}|p{6cm}| p{6cm}|}
    \hline
    \textbf{Requirement Number} & \textbf{Requirement Name} & \textbf{Requirement Text}\\
    \hline
    F7& Handle Configuration File Existence & The PFT shall handle the initial creation, checking for existence of and recreation of the configuration file in the event of its absence.  \\
    \hline
\end{longtable}

\subsection{Create Initial Configuration File In Default Path Following Initial Financial Information Collection}

\begin{longtable}{|p{2cm}|p{6cm}| p{6cm}|}
    \hline
    \textbf{Requirement Number} & \textbf{Requirement Name} & \textbf{Requirement Text}\\
    \hline
    F7.1 & Create Initial Configuration File In Default Path Following Initial Financial Information Collection & The PFT shall create the configuration file containing the entered initial financial information at a default path on the user's machine.  \\
    \hline
\end{longtable}

\subsection{Upon Subsequent Application Launches Check For Existence Of Configuration File}

\begin{longtable}{|p{2cm}|p{6cm}| p{6cm}|}
    \hline
    \textbf{Requirement Number} & \textbf{Requirement Name} & \textbf{Requirement Text}\\
    \hline
    F7.2 & Upon Subsequent Application Launches Check For Existence Of Configuration File & The PFT shall check for the existence of the configuration file on subsequent launches of the application.  \\
    \hline
\end{longtable}

\subsection{Generate A Default Configuration File At Default Path If Not Found On Subsequent Launches}

\begin{longtable}{|p{2cm}|p{6cm}| p{6cm}|}
    \hline
    \textbf{Requirement Number} & \textbf{Requirement Name} & \textbf{Requirement Text}\\
    \hline
    F7.3 & Generate A Default Configuration File At Default Path If Not Found On Subsequent Launches & The PFT shall generate a default configuration file at the default path on subsequent launches of the application if the file was not found.  \\
    \hline
\end{longtable}

\end{document}